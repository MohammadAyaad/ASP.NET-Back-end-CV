%%%%%%%%%%%%%%%%%
% This is an sample CV template created using altacv.cls
% (v1.1.2, 1 February 2017) written by LianTze Lim (liantze@gmail.com). Now compiles with pdfLaTeX, XeLaTeX and LuaLaTeX.
% 
%% It may be distributed and/or modified under the
%% conditions of the LaTeX Project Public License, either version 1.3
%% of this license or (at your option) any later version.
%% The latest version of this license is in
%%    http://www.latex-project.org/lppl.txt
%% and version 1.3 or later is part of all distributions of LaTeX
%% version 2003/12/01 or later.
%%%%%%%%%%%%%%%%

%% If you need to pass whatever options to xcolor
\PassOptionsToPackage{dvipsnames}{xcolor}

%% If you are using \orcid or academicons
%% icons, make sure you have the academicons 
%% option here, and compile with XeLaTeX
%% or LuaLaTeX.
% \documentclass[10pt,a4paper,academicons]{altacv}
\documentclass[10pt,a4paper]{altacv}

%% AltaCV uses the fontawesome and academicon fonts
%% and packages. 
%% See texdoc.net/pkg/fontawecome and http://texdoc.net/pkg/academicons for full list of symbols.
%% 
%% Compile with LuaLaTeX for best results. If you
%% want to use XeLaTeX, you may need to install
%% Academicons.ttf in your operating system's font 
%% folder.


% Change the page layout if you need to
\geometry{left=1cm,right=9cm,marginparwidth=6.8cm,marginparsep=1.2cm,top=1.25cm,bottom=1.25cm}

% Change the font if you want to.

% If using pdflatex:
\usepackage[utf8]{inputenc}
\usepackage[T1]{fontenc}
\usepackage[default]{lato}
\usepackage{hyperref}

% If using xelatex or lualatex:
% \setmainfont{Lato}

% Change the colours if you want to
\definecolor{Mulberry}{HTML}{72243D}
\definecolor{SlateGrey}{HTML}{2E2E2E}
\definecolor{LightGrey}{HTML}{666666}
\colorlet{heading}{Sepia}
\colorlet{accent}{Mulberry}
\colorlet{emphasis}{SlateGrey}
\colorlet{body}{LightGrey}

% Change the bullets for itemize and rating marker
% for \cvskill if you want to
\renewcommand{\itemmarker}{{\small\textbullet}}
\renewcommand{\ratingmarker}{\faCircle}

%% sample.bib contains your publications
\addbibresource{sample.bib}

\begin{document}
\name{Mohammad Ayaad}
\tagline{.NET Back-end Developer}
\personalinfo{%
  % Not all of these are required!
  % You can add your own with \printinfo{symbol}{detail}
  \email{mohammadayaad1024@gmail.com}
  \phone{+01102198815}
  \location{Cairo, Egypt}\\
  \linkedin{linkedin.com/in/mohammad-ayaad}
  \github{github.com/MohammadAyaad}
  \twitter{twitter.com/mohammad\_ayaad}
  \href{grepper.com/profile/mohammad-ayaad}{grepper.com/profile/mohammad-ayaad}\,\,\,\,\,\,\,\,
  \href{sololearn.com/en/profile/25776005}{sololearn.com/en/profile/25776005}\,\,\,\,\,\,
  \href{hackerrank.com/profile/ayaadm730}{hackerrank.com/profile/ayaadm730}\,\,\,\,\,\,\,\,
  %% You MUST add the academicons option to \documentclass, then compile with LuaLaTeX or XeLaTeX, if you want to use \orcid or other academicons commands.
%   \orcid{orcid.org/0000-0000-0000-0000}
}

%% Make the header extend all the way to the right, if you want. Extend the right margin by 8cm (=6.8cm marginparwidth + 1.2cm marginparsep)
\begin{adjustwidth}{}{-8cm}
\makecvheader
\end{adjustwidth}

%% Provide the file name containing the sidebar contents as an optional parameter to \cvsection.
%% You can always just use \marginpar{...} if you do
%% not need to align the top of the contents to any
%% \cvsection title in the "main" bar.
\cvsection[page1sidebar]{Experience}
\begin{itemize}
\item \textbf{Over 8 years} of experience in developing personal projects, with a broad range of applications. My projects span from low-level OS assembly code, through low-level graphics programming and ray tracing, to back-end programming and API creation.
\item \textbf{6+ years} of experience in using .NET for personal projects. This including but not limited to creating APIs, web applications, Games, Desktop Apps, and other software solutions using various .NET technologies such as C\# .NET, .NET Core, .NET Framework, ASP.NET Core, Blazor, and Razor.
\item \textbf{4+ years} of tutoring courses in various technologies. This role showcases my ability to explain complex concepts in an understandable manner and my passion for knowledge sharing.
\item \textbf{3+ years} of experience in low-level graphics programming through personal projects, demonstrating my ability to work close to the hardware and optimize for performance.
\item \textbf{3+ years} of experience in back-end development through personal projects, showcasing my skills in server-side logic, database interactions, and server performance optimization.
\item \textbf{2+ years} of successful freelancing, providing tailored software solutions to various clients across different industries. This experience has honed my problem-solving skills and ability to deliver high-quality work under tight deadlines.
\item Authored more than \textbf{10GiB} of source code, reflecting my hands-on experience and dedication to coding. This vast amount of code demonstrates my ability to handle large-scale projects and maintain high coding standards.
\item Experience with more than \textbf{10 different frameworks and libraries}, demonstrating my ability to choose the right tool for the job and quickly master new technologies. These include popular frameworks and libraries for web development and API creation.
\end{itemize}

\divider


\newpage

\cvsection[page2sidebar]{Projects}
\cvevent{ShopAPI: An Advanced E-Commerce Platform}{Personal Project}{June 2022 - April 2024}{}
\begin{itemize}
    \item Engineered an advanced API for an e-commerce platform, significantly enhancing functionality and optimizing user experience, leading to increased user engagement.
    \item Devised a dynamic product system capable of accommodating a wide range of e-commerce scenarios, demonstrating adaptability and forward-thinking in meeting diverse customer preferences.
    \item Innovated a package feature that enables product bundling, a strategic move that increased the value of customer purchases and boosted sales.
    \item Implemented a robust user account system, ensuring secure and reliable account management, enhancing user trust and satisfaction, and reducing security incidents.
    \item Integrated Swagger UI, providing comprehensive API documentation that improved developer usability and accelerated API integration, reducing development time.
    \item Utilized Docker for application containerization, ensuring consistent deployment across various environments and simplifying the development process, leading to increased productivity.
    \item Gained invaluable experience in API development, adeptly balancing technical requirements with business needs to deliver high-quality solutions, demonstrating a strong understanding of both technology and business.
\end{itemize}
\cvevent{JsonTokens: A Novel C\# Library for API Token Management}{Personal Project}{Jan 2024 - March 2024}{}
\begin{itemize}
    \item Developed JsonTokens, a unique C\# library that introduces a novel type of token for API use, showcasing my ability to innovate and create tailored solutions that address specific needs in the realm of web development.
    \item The library is built upon JSON, a widely-used data interchange format, demonstrating my deep understanding of industry standards and protocols, and my ability to apply them effectively in software development.
    \item One of the standout features of JsonTokens is its ability to store C\# objects. This not only exhibits my proficiency in object-oriented programming but also underscores my ability to design flexible and reusable software components that enhance code efficiency and maintainability.
    \item JsonTokens supports pre-processing steps that can be added between the point when the token is used on the API and when it is sent to the user. This feature highlights my foresight in anticipating potential use cases and my ability to design software with extensibility in mind, ensuring the library can adapt to future requirements with minimal modifications.
    \item This project served as a platform for me to further hone my skills in C\# and .NET, and deepen my understanding of API development and token management, preparing me for more complex and challenging projects in the future.
\end{itemize}


\divider
\newpage
\cvsection[page3sidebar]{More Projects}{}

\cvevent{GeTop API: A Robust Social Media Content Management Solution}{Personal Project}{October 2022 - April 2024}{}
\begin{itemize}
    \item Developed GeTop API, a robust and versatile API designed to handle a "social media-like" website, capable of managing a diverse range of content such as posts, videos, and images. This project showcases my ability to create comprehensive and multifunctional solutions that cater to complex business needs.
    \item The API supports storing images, demonstrating my proficiency in handling and managing multimedia content in web development, and my understanding of the technical requirements associated with image processing and storage.
    \item In addition to images, the API also supports storing videos, highlighting my ability to work with different media formats and my understanding of the technical requirements and challenges associated with video content management.
    \item GeTop API features an Accounts System with a Multi-Method Authentication System. This system allows for authenticating with multiple methods and serves as a platform for implementing Multi-Factor Authentication (MFA), showcasing my knowledge of advanced security measures and my ability to implement them in my projects.
    \item This project served as a platform for me to further hone my skills in C\# and .NET, deepen my understanding of API development and content management, and gain hands-on experience in implementing advanced security measures, preparing me for more complex and challenging projects in the future.
\end{itemize}

\divider

\cvevent{Contacts Manager: A Comprehensive Contact Management Solution}{University Project}{2024}{}
\begin{itemize}
    \item Collaborated with a team to develop Contacts Manager, a comprehensive application designed to manage contacts as part of a university project. This experience enhanced my teamwork and cooperation skills.
    \item The application supports storing a wide range of contact details, including first name, last name, job title, multiple phone numbers, and multiple email addresses, demonstrating my proficiency in handling and managing diverse data types in application development.
    \item Implemented a search algorithm that uses Levenshtein's Distance Algorithm, taking into account the length of the string. This showcases my ability to implement complex algorithms and my understanding of string manipulation techniques.
    \item Utilized SQLite for database management, reflecting my ability to work with different database systems and my understanding of SQL.
    \item The project was managed using Git and GitHub, demonstrating my familiarity with version control systems and collaborative coding platforms.
    \item This project served as a platform for me to further hone my skills in Java, deepen my understanding of application development and data management, and gain hands-on experience in implementing user-friendly interfaces and complex algorithms, preparing me for more complex and challenging projects in the future.
\end{itemize}




\clearpage

%% If the NEXT page doesn't start with a \cvsection but you'd
%% still like to add a sidebar, then use this command on THIS
%% page to add it. The optional argument lets you pull up the 
%% sidebar a bit so that it looks aligned with the top of the
%% main column.
% \addnextpagesidebar[-1ex]{page3sidebar}


\end{document}
